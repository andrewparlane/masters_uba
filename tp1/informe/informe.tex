\documentclass[a4paper]{article}

\usepackage[toc,page]{appendix}

\usepackage[spanish]{babel}
\usepackage[utf8]{inputenc}
\usepackage{amsmath}
\usepackage{graphicx}
\usepackage{listings}
\usepackage{fancyhdr}
\usepackage{amsmath}
\usepackage[colorinlistoftodos]{todonotes}
\usepackage{xcolor}
\usepackage{minted}
\usepackage{enumitem}
\usepackage{hyperref}

\hypersetup{
    colorlinks,
    citecolor=blue,
    filecolor=blue,
    linkcolor=blue,
    urlcolor=blue
}

\definecolor{mGreen}{rgb}{0,0.6,0}
\definecolor{mGray}{rgb}{0.5,0.5,0.5}
\definecolor{mPurple}{rgb}{0.58,0,0.82}
\definecolor{backgroundColour}{rgb}{0.95,0.95,0.92}



\usepackage{geometry}
 \geometry{
 a4paper,
 total={210mm,297mm},
 left=22mm,
 right=19.1mm,
 top=25.4mm,
 bottom=25.4mm,
 }

 \lstdefinestyle{CStyle}{
    backgroundcolor=\color{backgroundColour},
    commentstyle=\color{mGreen},
    keywordstyle=\color{magenta},
    numberstyle=\tiny\color{mGray},
    stringstyle=\color{mPurple},
    basicstyle=\footnotesize,
    breakatwhitespace=false,
    breaklines=true,
    captionpos=b,
    keepspaces=true,
    numbers=left,
    numbersep=5pt,
    showspaces=false,
    showstringspaces=false,
    showtabs=false,
    tabsize=2,
    language=C
}


\begin{document}
\begin{figure}
\centering
\includegraphics[scale=1]{./img/logo-facu}
\end{figure}

\title{\large\textsc{66.20 - Organización de Computadoras}\\
\large Trabajo Práctico 1 - Conjunto de instrucciones MIPS}

\author{
Nicolás Alvarez, Padrón 93503\\
Nicolás Fernandez, Padrón 88599\\
Andrew Parlane \hspace{2cm} \\
}

\maketitle

\newpage

\tableofcontents

\newpage

\section{Resumen}
El objetivo de este trabajo se basa en aplicar los conocimientos aprendidos en clase acerca de las instrucciones MIPS32 y el concepto de ABI.  Para eso desarrollaremos un programa en C que utilizará una función desarrollada en lenguaje assembler MIPS32.

\section{Introducción}
El programa que desarrollaremos se encargará de tomar una matriz de números enteros y devolverá su matriz transpuesta. El programa recibirá como argumento el nombre del archivo donde está especificada la matriz, y dará el resultado por stdout o lo escribirá en un archivo, según las opciones de entrada. De haber errores, los mensajes de error saldrán exclusivamente por stderr.

\section{Compilacion y ejecución del programa}
Esta sección describe  los mecanismos para la compilación y ejecución del programa, tanto en entorno MIPS como en Linux.

En una primera etapa compilamos y trabajamos en Linux para poder realizar las pruebas. Adicionalmente a esto utilizamos Valgrind para chequear que no tengamos problemas de leaks o free en nuestro código.\newline

Para compilar usando el Makefile:
\begin{itemize}
\item Estando parados en la carpeta donde se encuentran los archivos fuente, ejecutamos el siguiente comando:
\begin{minted}{bash}
host# make
\end{minted}
Y para la machina de MIPS:
\begin{minted}{bash}
host# gmake
host# gmake asm
\end{minted}
\end{itemize}

Para ejecutar:
\begin{itemize}
\item Estando parados en la carpeta donde se encuentran el archivo ejecutable, corremos para cada archivo o fragmento de prueba lo siguiente:
\begin{minted}{bash}
./tp1_c [options] file
\end{minted}
Y en la machina MIPS:
\begin{minted}{bash}
./tp1_asm [options] file
\end{minted}
\item Si se quiere ver el help, que especifica las opciones disponibles al momento de invocar el programa:
\begin{minted}{bash}
./tp1_c -h
\end{minted}
\item Si se quiere ver la versión del programa:
\begin{minted}{bash}
./tp1_c -V
\end{minted}
\end{itemize}

\section{Implementación}
\subsection{Implementación en C}

El trabajo se estructuró en tres archivos, uno que poseerá la mayor parte del programa (procesamiento de argumentos, usage, etc.) y lo otros dos la función \textbf{\textit{transponer}} en código C y en código assembler MIPS32:

\begin{itemize}
	\item \textbf{main.c} : Define el proceso principal de ejecución, la validación de los parámetros pasados al programa y además los métodos para parseo y salida del programa.
    \item \textbf{transpose.c} : Define la función transponer en código c, que recibe una matriz y devuelve su transpuesta.
    \item \textbf{transpose.S} : Define la función transponer en código assembler, que recibe una matriz y devuelve su transpuesta.
\end{itemize}
A continuación enumeramos las funciones definidas en el programa que se usarán luego de la verificación y validación de los parámetros de entrada:
\begin{itemize}
\item \textbf{usage}\\
	\begin{tabular}{ll}
    parámetros: &FILE *stream \\
    			&const char *nuestroNombre\\
    &\\
    descripción: &función que muestra el help de la aplicación.
    \end{tabular}\\

\item \textbf{leerLongLong}\\
	\begin{tabular}{ll}
    parámetros:    &FILE *f\\
                   &long long *ll\\
                   &bool *OK\\
                   &bool *eof\\
                   &bool *newLine\\
	&\\
    descripción: &función que lee un entero de la matriz.
    \end{tabular}\\

\item \textbf{leerLinea}\\
	\begin{tabular}{ll}
    parámetros:    &FILE *f\\
                   &long long *data\\
                   &uint columnasEsperados\\
                   &bool *eof\\
    &\\
    descripción: &función que lee una línea completa de la matriz.
    \end{tabular}\\

\item \textbf{leerEntrada}\\
	\begin{tabular}{ll}
    parámetros:     &const char *archivo\\
    				&uint *filas\\
                    &uint *columnas\\
    &\\
    descripción: &función que lee el archivo de entrada e inicia el procesamiento de la matriz.
    \end{tabular}\\
\item \textbf{escribirSalida}\\
	\begin{tabular}{ll}
    parámetros:     &const char *archivo\\
    				&uint filas\\
                    &uint columnas\\
                    &long long *salida\\
    &\\
    descripción: &función que escribe el resultado a un archivo o stdout.
    \end{tabular}\\
\item \textbf{transponer}\\
	\begin{tabular}{ll}
    parámetros:     &unsigned int *filas\\
                    &uint *columnas\\
                    &long long *entrada\\
                    &long long *salida\\
    &\\
    descripción: &función lee una matriz y devuelva su matriz transpuesta.
    \end{tabular}\\
\end{itemize}
\lstset{basicstyle=\footnotesize\ttfamily}

\section{Pruebas}
Realizamos las pruebas en GXEmul para cada uno de los archivos pedidos.
\setlist{nosep}
\begin{itemize}
\item \textbf{matrix1}
\item \textbf{matrix2}
\item \textbf{matrix3}
\end{itemize}

\begin{minted}{bash}
$ ./tp1_c -o - pruebas/matrix1
7 1
1
2
3
4
5
6
7

$ ./tp1_c -o - pruebas/matrix2
Not enough entries on line. Expecting 5, found 4

$ ./tp1_c -o - pruebas/matrix3
Found invalid character .

\end{minted}

También se realizó unas pruebas con otros archivos para detectar otros casos posibles en el archivo de entrada.

\begin{minted}{bash}
$ cat matrix_tabs
4   3
1   2   3
		4	   5	   6
7 8 9
10					  11 12

$ ./tp1_c -o - pruebas/matrix_tabs
3 4
1 4 7 10
2 5 8 11
3 6 9 12

$ cat matrix_negativo
4 2
2 1
0 -1
-2 -3
-4 -5

$ ./tp1_c -o - pruebas/matrix_tabs
2 4
2 0 -2 -4
1 -1 -3 -5

$ cat matrix_long_long
5 4
9223372036854775807 0 1234567891011121314 1
1516171819202122232 4252627282930313233 2 3
4 3435363738394041424 3444546474849505152 6
5354555657585960616 2636465666768697071 7273747576777787980 8182838485868788899
0 0 0 0

$ ./tp1_c -o - pruebas/matrix_long_long
4 5
9223372036854775807 1516171819202122232 4 5354555657585960616 0
0 4252627282930313233 3435363738394041424 2636465666768697071 0
1234567891011121314 2 3444546474849505152 7273747576777787980 0
1 3 6 8182838485868788899 0
\end{minted}

Se incluirán en la entrega más archivos que fueron usados para probar la robustez del programa. También el Makefile incluye una regla "prueba" que ambos de tp1\_c y tp1\_asm por cada prueba, comparando el código de salida con lo que es esperado, y el matrix resultado con lo que es esperado.

\section{Diagrama del stack del programa}
A continuación se podrá ver el diagrama de cómo quedaría el stack justo después de entrando la función transponer. Usando \textit{objdump} encontramos que el stack de main es 88 bytes con 16 bytes por la SRA. La función \textbf{\textit{leerLongLong}} tiene cinco argumentos, así la ABA de main debería estar 24 bytes. Los último 48 bytes están la LTA.
\begin{minted}{bash}
#   +---------------+   |------------MAIN------------
# 92    |       |   |   \
#   +---------------+   |
# 88    |   ra  |   |
#   +---------------+   |   SRA MAIN
# 84    |   S8  |   |
#   +---------------+   |
# 80    |   gp  |   |   /
#   +---------------+
# 76    |  LTA  |   |   \
#   +---------------+   |
# ..    |  LTA  |   |
#   +---------------+   |   LTA MAIN
# ..    |  LTA  |   |
#   +---------------+   |
# 32    |  LTA  |   |   /
#   +---------------+
# 28    |       |   |   \
#   +---------------+   |
# 24    |       |   |
#   +---------------+   |
# 20    |salida |   |
#   +---------------+   |   ABA MAIN
# 16    |entrada|   |
#   +---------------+   |
# 12    |columnas|  |
#   +---------------+   |
# 08    | filas |   |   /
#   +---------------+   |----------TRANSPONER----------
# 04    |   fp  |   |   \
#   +---------------+   |   SRA TRANSPONER
# 00    |   gp  |   |   /
#   +---------------+
\end{minted}


\section{Conclusiones}
	El desarrollo de este trabajo nos permitió llevar a la práctica los conocimientos adquiridos acerca de la estructura MIPS32.  Debimos respetar la ABI de esta arquitectura, respetando los tamaños de las distintas secciones: SRA (Saved Register Area), LTA (Local and Temp Area) y ABA (Arg Building Area).

    Para otorgar portabilidad a esta arquitectura desarrollamos la función transponer en lenguaje assembler MIPS32. Pudimos comprabar que fue un éxito al realizar la compilación de programa con la función \textbf{\textit{transponer}} desarrollada en assembler en el archivo \textbf{\textit{transpose.S}}

    Una observación que se puede apreciar en el diagrama del stack es que la función transponer posee un stack de 8 bytes ya que es una función \textit{hoja} por lo cual no tendrá una ABA ni una LTA y no se deberá guardar el registro RA, mientras que el stack del main mide 88 bytes ya que reserva espacio de ABA para los argumentos que se usarán en llamar otras funciones, y como no es una función \textit{hoja} guarda algunas variables temporales en la LTA.

\section{Código}

\subsection{main.c}
\begin{minted}{C}
#include <stdio.h>
#include <stdlib.h>
#include <libgen.h>
#include <getopt.h>
#include <ctype.h>
#include <string.h>
#include <stdint.h>
#include <stdbool.h>
#include <errno.h>

#define MAJOR_VERSION   1
#define MINOR_VERSION   0

/*
 *   +---------------+   |------------MAIN------------
 * 84    |       |   |   \
 *   +---------------+   |
 * 80    |   ra  |   |
 *   +---------------+   |   SRA MAIN
 * 76    |   S8  |   |
 *   +---------------+   |
 * 72    |   gp  |   |   /
 *   +---------------+
 * 68    |  LTA  |   |   \
 *   +---------------+   |
 * ..    |  LTA  |   |
 *   +---------------+   |   LTA MAIN
 * ..    |  LTA  |   |
 *   +---------------+   |
 * 24    |  LTA  |   |   /
 *   +---------------+
 * 20    |  ABA  |   |   \
 *   +---------------+   |
 * 16    |  ABA  |   |
 *   +---------------+   |
 * 12    |  ABA  |   |
 *   +---------------+   |   ABA MAIN
 *  8    |  ABA  |   |
 *   +---------------+   |
 *  4    |  ABA  |   |
 *   +---------------+   |
 *  0    |  ABA  |   |   /
 *   +---------------+
 */

// declaración adelante
// transponer es en transponer.c o transponer.s
extern int transponer(unsigned int filas,
                      unsigned int columnas,
                      long long *entrada,
                      long long *salida);

static const struct option long_options[] =
{
    {"help",    no_argument,        0, 'h' },
    {"version", no_argument,        0, 'V' },
    {"output",  required_argument,  0, 'o' },
    {0,         0,                  0,  0  }
};

static void usage(FILE *stream, const char *nuestroNombre)
{
    fprintf(stream,
           "Usage:\n"
           "  %s -h\n"
           "  %s -V\n"
           "  %s [options] filename\n"
           "Options:\n"
           "  -h, --help Prints usage information.\n"
           "  -V, --version Prints version information.\n"
           "  -o, --output Path to output file.\n"
           "\n"
           "Examples:\n"
           "  %s -o - mymatrix\n",
           nuestroNombre, nuestroNombre, nuestroNombre, nuestroNombre);
    // necesitamos usar nuestroNombre, nuestroNombre, nuestroNombre, nuestroNombre
    // porque no soportamos %1$s
}

// lea carácter por carácter deshaciendo whitespace
// hasta encontrar [0-9-]. Después comenzar a leer números
// [0-9]. Para cuando obtenemos EOF, \n, \r, ' ', \t.
// Es un error si encontramos algún otro carácter.
// devolver no 0 si hay un error
// *OK = 1 -> hay un integer valido en *ll
// *eof = 1 -> no hay más a leer
// *newLine = 1 -> encontramos nueva línea
static bool leerLongLong(FILE *f, long long *ll, bool *OK, bool *eof, bool *newLine)
{
    *OK = false;
    *eof = false;
    *newLine = false;

    // soportamos signed 64 bits:
    // máx = 0x7FFF_FFFF_FFFF_FFFF =  9223372036854775807
    // min = 0x8000_0000_0000_0000 = -9223372036854775808
    // así max input legal es 20 cáracters +1 por NULL terminator
#define MAX_CHARS 20
    char buff[MAX_CHARS + 1];
    uint idx = 0;

    bool comenzandoLeerInt = false;
    while (1)
    {
        int res = fgetc(f);
        if (res == EOF)
        {
            *eof = true;
            // si tenemos algo en buff, convertimos ahora
            if (*OK)
            {
                buff[idx] = '\0';
                *ll = strtoll(buff, NULL, 10);
                if (errno != 0)
                {
                    fprintf(stderr, "Failed to convert %s to long long, error: %s\n", buff, strerror(errno));
                    return false;
                }
                return true;
            }
            else if (comenzandoLeerInt)
            {
                // solo podríamos estar aquí si leemos
                // un '-' y después nada, eso es un error
                fprintf(stderr, "Found invalid entry \"-\"\n");
                return false;
            }
            else
            {
                // eof but no error
                return true;
            }
        }

        char c = (char)res;
        if (c == '\r' || c == '\n')
        {
            *newLine = true;
        }

        if (!comenzandoLeerInt)
        {
            // Todavía no cemenzamos a leer el int
            if (c == ' ' || c == '\t')
            {
                // ignoramos
                continue;
            }
            else if (*newLine)
            {
                // nuevo línea, terminamos.
                return true;
            }
            else if (c >= '0' && c <= '9')
            {
                // válido
                buff[idx++] = c;
                comenzandoLeerInt = true;
                *OK = true;
            }
            else if (c == '-')
            {
                // también válido pero el int todavía no es OK
                // porque necesitamos un número después de un -
                buff[idx++] = c;
                comenzandoLeerInt = true;
            }
            else
            {
                // error
                fprintf(stderr, "Found invalid character %c\n", c);
                return false;
            }
        }
        else
        {
            // ya estamos leyendo data
            if (c == ' ' || c == '\t' ||
                *newLine)
            {
                // terminamos
                if (*OK)
                {
                    buff[idx] = '\0';
                    *ll = strtoll(buff, NULL, 10);
                    if (errno != 0)
                    {
                        fprintf(stderr, "Failed to convert %s to long long, error: %s\n", buff, strerror(errno));
                        return false;
                    }
                    return true;
                }
                else
                {
                    // solo podríamos estar aquí si leemos
                    // un '-' y después nada, eso es un error
                    fprintf(stderr, "Found invalid entry \"-\"\n");
                    return false;
                }
            }
            else if (c == '-')
            {
                // un - aquí no es válido porque estámos en el medio
                // de un int.
                fprintf(stderr, "Found \"-\" in the middle of an integer\n");
                return false;
            }
            else if (c >= '0' && c <= '9')
            {
                // válido
                if (idx >= MAX_CHARS)
                {
                    fprintf(stderr, "Integer read was too large to fit into a long long\n");
                    return false;
                }
                buff[idx++] = c;
                *OK = true;
            }
            else
            {
                // error
                fprintf(stderr, "Found invalid character %c\n", c);
                return false;
            }
        }
    }
}

static bool leerLinea(FILE *f, long long *data, uint columnasEsperados, bool *eof)
{
    uint32_t count = 0;
    while (1)
    {
        bool OK;
        bool newLine;
        long long ll;
        if (!leerLongLong(f, &ll, &OK, eof, &newLine))
        {
            // error
            return false;
        }

        if (OK)
        {
            // leemos un integer
            if (count >= columnasEsperados)
            {
                // error - hay mas columnas de las esperadas
                fprintf(stderr, "Too many entries on line. Expecting %u\n", columnasEsperados);
                return false;
            }

            data[count] = ll;
            count++;
        }

        if (*eof)
        {
            if (count != columnasEsperados)
            {
                // error - hay menos columnas de las esperadas
                fprintf(stderr, "Not enough entries on line. Expecting %u, found %u\n", columnasEsperados, count);
                return false;
            }
            else
            {
                return true;
            }
        }

        if (newLine)
        {
            if (count == 0) // permitimos newLines antes de data comenzando
            {
                continue;
            }
            else if (count != columnasEsperados)
            {
                // error - hay menos columnas de las epseradas.
                fprintf(stderr, "Not enough entries on line. Expecting %u, found %u\n", columnasEsperados, count);
                return false;
            }
            else
            {
                return true;
            }
        }
    }
    return true;
}

static long long *leerEntrada(const char *archivo, uint *filas, uint *columnas)
{
    FILE *f = fopen(archivo, "r");
    if (f == NULL)
    {
        fprintf(stderr, "%s: No such file or directory\n", archivo);
        return NULL;
    }

    long long primerLinea[2];

    bool eof;
    if (!leerLinea(f, primerLinea, 2, &eof))
    {
        fclose(f);
        return NULL;
    }

    long long *llFilas = &primerLinea[0];
    long long *llColumnas = &primerLinea[1];

    // Validar filas y columnas
    // no pueden ser menor a cero
    // ni mas grande que 0xFFFFFFFF
    if (*llFilas < 0 || *llColumnas < 0 ||
        *llFilas > 0xFFFFFFFF ||
        *llColumnas > 0xFFFFFFFF)
    {
        fprintf(stderr, "Invalid number of rows / columns\n");
        fclose(f);
        return NULL;
    }

    *filas = *(uint *)llFilas;
    *columnas = *(uint *)llColumnas;

    // numero de elementos = filas * columnas
    // cada uno es un long long, así:
    long long *entrada = malloc(*filas * *columnas * sizeof(long long));
    if (entrada == NULL)
    {
        fprintf(stderr, "Failed to malloc %u bytes\n", (unsigned int)(*filas * *columnas * sizeof(long long)));
        fclose(f);
        return NULL;
    }

    uint i;
    for (i = 0; i < *filas; i++)
    {
        if (!leerLinea(f, &entrada[i * *columnas], *columnas, &eof))
        {
            fclose(f);
            free(entrada);
            return NULL;
        }
    }

    // debería estar todo, comprobar que no hay más data
    while (!eof)
    {
        if (!leerLinea(f, NULL, 0, &eof))
        {
            fclose(f);
            free(entrada);
            return NULL;
        }
    }

    fclose(f);
    return entrada;
}

static bool escribirSalida(const char *archivo, uint filas, uint columnas, long long *salida)
{
    FILE *f;

    if (archivo == NULL)
    {
        // stdout
        f = stdout;
    }
    else
    {
        // archivo
        f = fopen(archivo, "w");
        if (f == NULL)
        {
            fprintf(stderr, "Failed to open %s for writing\n", archivo);
            return NULL;
        }
    }

    fprintf(f, "%u %u\n", filas, columnas);
    uint i;
    for (i = 0; i < filas; i++)
    {
        uint c;
        for (c = 0; c < columnas; c++)
        {
            fprintf(f, "%lld ", salida[(i * columnas) + c]);
        }
        fprintf(f, "\n");
    }

    if (archivo != NULL)
    {
        fclose(f);
    }

    return true;
}

int main(int argc, char **argv)
{
    // usamos argv[0] como el nombre del aplicación
    // pero solo queremos el archivo, no la ruta
    const char *nuestroNombre = basename(argv[0]);

    // escribir la salida a un archivo si vemos -o (y el argumento no es -)
    const char *oArchivo = NULL;

    // clear errors
    opterr = 0;

    // parse short options
    while (1)
    {
        // obtener el siguiente argumento
        int option_index = 0;
        int c = getopt_long(argc, argv, "hVo:", long_options, &option_index);

        if (c == -1)
        {
            // no hay más
            break;
        }

        switch (c)
        {
            case 'h':
            {
                usage(stdout, nuestroNombre);
                // no seguimos despues de -h
                return 0;
            }
            case 'V':
            {
                printf("%s: Version %u.%u\n", nuestroNombre, MAJOR_VERSION, MINOR_VERSION);
                // no seguimos despues de -V
                return 0;
            }
            case 'o':
            {
                // si vemos "-o -" la salida es stdout
                // si no, la salida es el archivo en optarg
                if (strcmp(optarg, "-") != 0)
                {
                    oArchivo = optarg;
                }
                break;
            }
            case '?':
            {
                if (optopt == 'o')
                {
                    fprintf(stderr, "Option '-%c' requires an argument.\n\n", optopt);
                }
                else if (isprint(optopt))
                {
                    // es un argumento, pero no es uno que esperamos
                    fprintf (stderr, "Unknown option '-%c'.\n\n", optopt);
                }
                else
                {
                    // solo muestra el usage
                }
                usage(stderr, nuestroNombre);
                return 1;
            }
            default:
            {
                usage(stderr, nuestroNombre);
                return 1;
            }
        }
    }

    if (optind == argc)
    {
        fprintf(stderr, "filename is required\n\n");
        usage(stderr, nuestroNombre);
        return 1;
    }

    if ((optind + 1) != argc)
    {
        fprintf(stderr, "Too many arguments\n\n");
        usage(stderr, nuestroNombre);
        return 1;
    }

    // leer archivo
    uint filas;
    uint columnas;
    long long *entrada = leerEntrada(argv[optind], &filas, &columnas);
    if (entrada == NULL)
    {
        // falla, leerArchivo escribí el error
        return 1;
    }

    // malloc la salida
    long long *salida = malloc(filas * columnas * sizeof(long long));
    if (salida == NULL)
    {
        fprintf(stderr, "Failed to allocate %u bytes for output\n", (unsigned int)(filas * columnas * sizeof(long long)));
        free(entrada);
        return 1;
    }

    // transponer
    if (transponer(filas, columnas, entrada, salida) != 0)
    {
        fprintf(stderr, "Failed to transpose the matrix\n");
        free(entrada);
        free(salida);
        return 1;
    }

    // escribir el resultado
    bool res = escribirSalida(oArchivo, columnas, filas, salida);
    free(entrada);
    free(salida);

    return res ? 0 : 1;
}
\end{minted}

\subsection{transpose.c}
\begin{minted}{C}
#include <stdio.h>
#include <stdint.h>
#include <stdlib.h>

int transponer(unsigned int filas, unsigned int columnas, long long *entrada, long long *salida)
{
    uint f;
    for (f = 0; f < filas; f++)
    {
        uint c;
        for (c = 0; c < columnas; c++)
        {
            salida[(c * filas) + f] = entrada[(f * columnas) + c];
        }
    }

    return 0;
}
\end{minted}

\subsection{transpose.S}
\begin{minted}{asm}
 #   +---------------+
 # 20    |salida |   |   \
 #   +---------------+   |
 # 16    |entrada|   |
 #   +---------------+   |   ABA MAIN
 # 12    |columnas|  |
 #   +---------------+   |
 # 08    | filas |   |   /
 #   +---------------+   |----------TRANSPONER----------
 # 04    |   fp  |   |   \
 #   +---------------+   |   SRA TRANSPONER
 # 00    |   gp  |   |   /
 #   +---------------+

#include <mips/regdef.h>

    .text
    .align  2
    .globl  transponer
    .ent    transponer

# int transponer(unsigned int filas,
#                unsigned int columnas,
#                long long *entrada,
#                long long *salida)

transponer:
    subu    sp, sp, 8
    sw      gp,  0(sp)
    sw      $fp, 4(sp)
    sw      a0,  8(sp)
    sw      a1, 12(sp)
    sw      a2, 16(sp)
    sw      a3, 20(sp)

    # a0 = filas
    # a1 = columnas
    # a2 = &entrada[0]
    # a3 = &salida[0]

    move    v0, zero            # siempre devolvemos 0 no hay errores posibles

    beqz    a0, fin             # if (filas == 0) return 0;
    beqz    a1, fin             # if (columnas == 0) return 0;

    move    t0, zero            # uint f = 0;

filaLoop:                       # do {
    move    t1, zero            #   uint c = 0;

columnaLoop:                    #   do {
    ld      t2, 0(a2)           #     (t2,t3) = *entrada;
    addu    a2, a2, 8           #     entrada++;

    mul     t4, t1, a0          #     t4 = (c * filas);
    addu    t4, t4, t0          #     t4 += f;
    mul     t4, t4, 8           #     t4 = offset en salida
    addu    t4, t4, a3          #     t4 = &salida[(c * filas) + f]
    sd      t2, 0(t4)           #     salida[(c * filas) + f] = (t2,t3)

    addu    t1, t1, 1           #     c++;
    bne     t1, a1, columnaLoop #   } while (c != columnas)

    addu    t0, t0, 1           #   f++;
    bne     t0, a0, filaLoop    # } while (f != filas)

fin:
    lw      gp, 0(sp)
    lw      $fp, 4(sp)
    addu    sp, sp, 8
    jr      ra

    .end transponer
\end{minted}

\subsection{Makefile}

\begin{minted}{Makefile}
C_TARGET = tp1_c
ASM_TARGET = tp1_asm

MACHINE = $(shell uname -m)

LIBS =
CC = gcc
C_FLAGS = -Wall -g
ASM_FLAGS = -Wall -g

default: $(C_TARGET)
all: default

C_OBJECTS = main.c.o \
		    transpose.c.o

ASM_OBJECTS = main.c.o \
			  transpose.S.o

HEADERS = $(wildcard *.h)

ifeq ($(MACHINE), pmax)

define HACE_PRUEBA_ASM
	@echo Probando código ASM con pruebas/$(strip $(1))
	@./pruebaScript.sh $(ASM_TARGET) $(strip $(1)) $(2)
endef

else

define HACE_PRUEBA_ASM
	@#Hace nada porque no estamos pmax
endef

endif

# Macro con dos argumentos
#  1) El nombre de archivo a probar en pruebas/
#  2) Código de salida esperado
#
# Si el código de salida es igual al esperada,
# combrobamos la salida con un archivo que
# tiene el mismo nombre del input en resultados/

define HACE_PRUEBA

	@echo Probando código C con pruebas/$(strip $(1))
	@./pruebaScript.sh $(C_TARGET) $(strip $(1)) $(2)
	$(call HACE_PRUEBA_ASM, $(1), $(2))
endef


%.c.o: %.c $(HEADERS)
	$(CC) $(C_FLAGS) -c $< -o $@

%.S.o: %.S $(HEADERS)
	$(CC) $(ASM_FLAGS) -c $< -o $@

$(C_TARGET): $(C_OBJECTS)
	$(CC) $(C_OBJECTS) $(C_FLAGS) $(LIBS) -o $@

$(ASM_TARGET): $(ASM_OBJECTS)
	$(CC) $(ASM_OBJECTS) $(ASM_FLAGS) $(LIBS) -o $@

C: $(C_TARGET)

ifeq ($(MACHINE), pmax)

ASM: $(ASM_TARGET)

else

ASM:

endif

c: C

asm: ASM

prueba: C ASM
	@# Primero el básico
	-$(call HACE_PRUEBA, matrix1, 0)
	@# Con finales de líneas diferentes
	-$(call HACE_PRUEBA, matrix_crlf, 0)
	-$(call HACE_PRUEBA, matrix_cr_only, 0)
	-$(call HACE_PRUEBA, matrix_lf_only, 0)
	@# Espacio blanco extra
	-$(call HACE_PRUEBA, matrix_filas_blancas, 0)
	-$(call HACE_PRUEBA, matrix_tabs, 0)
	@# Numeros negativos o grandes, pero en el rango de signed long long
	-$(call HACE_PRUEBA, matrix_negativo, 0)
	-$(call HACE_PRUEBA, matrix_long_long, 0)
	@# Inválidos
	-$(call HACE_PRUEBA, matrix2, 1)
	-$(call HACE_PRUEBA, matrix3, 1)
	-$(call HACE_PRUEBA, matrix_filas_negativos, 1)
	-$(call HACE_PRUEBA, matrix_demasiado_columnos, 1)
	-$(call HACE_PRUEBA, matrix_demasiado_filas, 1)
	-$(call HACE_PRUEBA, matrix_demasiado_largo_int, 1)
	-$(call HACE_PRUEBA, matrix_demasiado_largo_int2, 1)
	-$(call HACE_PRUEBA, matrix_demasiado_negativo_int, 1)
	-$(call HACE_PRUEBA, matrix_demasiado_negativo_int2, 1)
	-$(call HACE_PRUEBA, matrix_no_suficiente_filas, 1)
	-$(call HACE_PRUEBA, archivo_que_no_existe, 1)

clean:
	-rm -f *.o
	-rm -f $(C_TARGET)
	-rm -f $(ASM_TARGET)
	-rm -f salida stdout stderr

.PHONY: default all C c ASM asm clean
\end{minted}

\subsection{pruebaScript.sh}
\begin{minted}{shell}
#!/bin/sh

# Argumentos:
#  $1 applicación
#  $2 entrada en pruebas/ sin la ruta
#  $3 código de salida esperada

./$1 -o salida pruebas/$2 > stdout 2> stderr
export RES_CODE=$?
if [ $RES_CODE -eq $3 ]; then
    if [ $RES_CODE -ne 0 ]; then
        echo "  OK";
    else
        diff -w salida pruebas/esperados/$2 > /dev/null;
        if [ $? -eq 0 ]; then
            echo "  OK";
        else
            echo "  Transpuesta no es como esperada";
        fi
    fi
else
    echo "  Código de salida no es esperado";
fi
\end{minted}

\section{Enunciado}
\textit{*Ver hojas anexadas}
\end{document}
